\chapter{Finite Element Methods}

Finite element methods are commonly used to solve elliptic PDEs for their geometric flexibility and ability to handle PDEs. The differences between FEM and FDM are :

\begin{itemize}
	\item The most attractive feature of FEM is its ability to handle complicated geometries(boundaries). FDM is restricted to handle rectangular shapes and simple alterations.
	\item The most attractive feature of FDM is that it is straightforward to implement.
	\item One could consider FDM a particular case of FEM approach.
\end{itemize}

We start with introducing some common definitions in FEM.

A \emph{triangle} consists of the \emph{interior}, the \emph{edges} and the \emph{vertices}. An edge connects each pair of vertices. The edge does not contain the vertices themselves, which we indicate by writing $(a, b)$ instead of $[a, b]$. This convention has the consequence that if $d \in e$ is any point on the edge, then $d$ is not a vertex of the triangle. The interior is all the triangle except the vertices and the edges, in other words, the face of the triangle.

A \emph{triangulation} of $\Omega$, denoted by $\tau$, is a set of triangles so that the interiors are disjoint and every point is in the closure of some triangle. A valid triangulation is one without slave nodes, which is defined as a vertex of one triangle that is on an edge of another triangle.

A \emph{piecewise linear} $C_0$ function is a continuous function $\mathcal{R}^2 \to \mathcal{R}$ defined on $\Omega$, that is affine(linear) when restricted to any $T \in \tau$. Here by affine we mean there are constants so that $u(x, y) = \alpha x + \beta y + \gamma$. And by piecewise affine we mean that for every $T \in \tau$, there exist $\alpha_T$, $\beta_T$, $\gamma_T$ such that $u(x, y) = \alpha_Tx + \beta_Ty + \gamma_T, \forall (x, y) \in T$. The PLT space, denoted by $\mathcal{S}_{\tau}$, consists of all piecewise linear functions on $\tau$ that satisfy the Dirichlet boundary conditions.